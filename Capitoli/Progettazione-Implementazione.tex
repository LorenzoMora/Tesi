\chapter{Progettazione}\label{chapter:progettazione}
Il seguente è lo pseudo codice che guida l'algoritmo della mia ricerca locale

\begin{verbatim}
    parser = PDBParser(PERMISSIVE = True, QUIET = True)
    title = '6vxx_chiusa.pdb'
    chain_of_interest = 'B'
    idx_chain_of_interest = -1
    covid_chiusa = parser.get_structure('STS', './covid_pdb/'+title)
    covid_loop = parser.get_structure('STS', './covid_pdb/6vxx_chiusa_loop.pdb')
    model = covid_loop.get_models()
    models = list(model)

    ### Parte di identificazione dei loop
    for model_chain in models:
        list_chain_loop = list(model_chain.get_chains())
        for id in range(len(list_chain_loop)):
            if (list_chain_loop[id].id == chain_of_interest):
                idx_chain_of_interest = id
        coil = []
        loop = identify_loop(list_chain_loop, idx_chain_of_interest, costanti_loopA)
        coil.append(loop)
        loop = identify_loop(list_chain_loop, idx_chain_of_interest, costanti_loopB)
        loop_r = loop[::-1]
        coil.append(loop_r)


    coil_dict_coordinates = [{"loop": x, "residui":[{"residuo":y, "coordinate": []}
                for y in range(len(coil[coil_name.index(x)]))]}for x in coil_name]
    for loop,idx in zip(coil, range(len(coil))):
        set_coordinate_loop(loop, idx)
\end{verbatim}
Innanzitutto viene creato attreverso la funzione messa a disposizione dalla libreria biopython il parser per poter effettivamente eseguire il parsing del file pdb.
Il file pdb è un formato particolare che viene utilizzato per il salvataggio su file delle strutture proteiche. Il parser generato in precedenza ci permette di
utilizzare alcuni metodi specifici come quello per ottenere la struttare proteica contenuta all'interno del file. La struttura a sua volta ha una composizione pressoche
ad albero e ci sono vari livelli al suo interno, noi prendiamo il modello e le catene all'interno del modello. La catena di nostro interesse è la B, una volta identificato
il suo id possiamo farci accesso come se stessimo indicizzando un array. 

Il prossimo step è quello di andare a recuperare i loop che collegano la parte fissa alla parte mobile utilizzando la procedura identify\_loop. I loop all'interno di
questa struttura sono due quindi abbiamo due costanti che contengono appunto le costanti di partenza e fine dei due loop. I due loop contengono due residui in più; 
questo, perché il loop A continente l'ultimo residuo della parte fissa all'inizio e il primo residuo della parte mobile alla fine, mentre il loop B contiene 
l'ultimo residuo della parte mobile all'inizio e il primo residuo della parte fissa alla fine. Essi hanno senso contrario, questo comporta che il loop B una volta
individuato venga invertito, ulteriormente invertito una volta che deve essere caricato come risultato. Vengono conservati due residui in più perché vogliamo essere
certi di attaccarci nel posto corretto e senza sbagliare ne l'aggancio con il punto di partenza nel con quello di arrivo per non rischiare di modificare la struttura.

\begin{verbatim}
    def identify_loop(list_chain, chain_of_interest, costanti):
        loop = []
        residue = list(list_chain[chain_of_interest].get_residues())
        for r in residue:
            if (r.id >= costanti[0]) and (r.id <= costanti[1]):
                loop.append(r)  
        return loop
\end{verbatim}
La procedura identify\_loop si occupa appunto di andare ad individuare i loop all'interno della catena. Vengono presi come parametro la lista delle catene, l'idx della
catena di interesse e le costanti. Una volta presa la catena di interesse abbiamo un metodo che ci permette di ottenere i residui all'interno della stessa e scorrendo 
i residui troviamo quelli che hanno l'id compreso tra le costanti. Viene restituita quindi la lista contenenti i residui che soddisfano la condizione.
\\
Una volta individuati i loop, salviamo le coordinate dei singoli atomi in una struttura creata apposta in modo tale da non aver problemi di condivisione della memoria,
ovvero non vogliamo che una modifica alla coordinata di un atomo sia condivisa con tutti gli altri atomi. 
\begin{verbatim}
    def set_coordinate_loop(loop, idx):
        appoggio = coil_dict_coordinates[idx]['residui']
        idx_res = 0
        for residue in loop:
            appoggio[idx_res]['coordinate'] = []
            for atom in residue.get_atoms():
                array = []
                c = atom.get_vector()
                for el in c.get_array():
                    array.append(el)
                appoggio[idx_res]['coordinate'].append(array)
            idx_res += 1
\end{verbatim}
La procedura set\_coordinate\_loop è molto semplice ovvero prende tutti i residui presenti all'interno dell loop, di ogni residuo prendiamo gli atomi e le relative 
coordinate [x, y, z] e le salviamo all'interno dell'apposito spazio creato per loro all'interno del dizionario.


\begin{verbatim}
    list_chain = list(models[0].get_chains())
    residui_parte_mobile = identificazione_parte_mobile(list_chain, 
                            idx_chain_of_interest)
    residui_parte_mobile_after_rotation = apply_rotation_on_mobil_part
                                (residui_parte_mobile)
    if (transition_step != 0):
        print("Applico la funzione di traslazione")
        applytransitionmove(residui_parte_mobile_after_rotation)
    res_pt_mob_copy = deepcopy(residui_parte_mobile_after_rotation)
    insert_new_mobil_part(list_chain, idx_chain_of_interest, 
                residui_parte_mobile_after_rotation)
    printpdb("Rotazione_parte_mobile.pdb")
    for loop, idx in zip(coil, range(len(coil))):
        print("Loop-",coil_name[idx])
        distanza_iniziale = objective_function(loop, 
            residui_parte_mobile_after_rotation, coil_name[idx])
        distanza_migliore.append(distanza_iniziale)
        distanza_raggiunta.append(distanza_iniziale)   
        angolo_positivo = False
        residuo_piccolo = False
        angolo_positivo_residuo = 0
        swap_residuo = 0 
        loop_new = []  
        print("Initial distance: ", distanza_iniziale)
        
        print("Rotazioni canoniche sulla meta del loop")
        for a in angle_to_direction:
            for j in [0,1]:
                if (coil_name[idx] == 'A'):
                    prova = hillclimbing(loop, [7, 8, 9, 10], a, 
                                        int(len(loop)/2)+1, j)
                    residui_usati.append(int(len(loop)/2)+1)
                else:
                    prova = hillclimbing(loop, [], a, 
                                        int(len(loop)/2), j)
                    residui_usati.append(int(len(loop)/2))
                distance = objective_function(prova, 
                        residui_parte_mobile_after_rotation, coil_name[idx])
                distanza_raggiunta.append(distance)
                angoli_usati.append(a)
                if (distance < distanza_iniziale):
                    distanza_iniziale = distance
                    distanza_migliore.append(distanza_iniziale)
                    set_coordinate_loop(prova, idx)
                get_coordinate_loop(loop, idx)
        print("Distanza dopo prova delle rotazioni sulla meta del loop: ", 
                            distanza_iniziale)

        if (coil_name[idx] == 'A'):
            insert_new_loop(list_chain, idx_chain_of_interest, loop, 
                    costanti_loopA, coil_name[idx])
        else:
            loop_af = loop[::-1] 
            insert_new_loop(list_chain, idx_chain_of_interest, loop_af, 
                    costanti_loopB, coil_name[idx])
        title = "Canonial_rotation" + coil_name[idx] +".pdb"
        printpdb(title)

        for i in range(iterazioni_massime):
            if (coil_name[idx] == 'A'):
                n_res = suggestedresidue(len(loop), distanza_iniziale, 
                                    distanza_raggiunta)
                angle = suggestedangle(distanza_iniziale, distanza_raggiunta)
                swap = suggestedchange(distanza_iniziale, distanza_raggiunta)
                if(swap):
                    contatore_swap_effettuati += 1
                    if angolo_positivo:
                        angle = 0.5
                    else:
                        angle = -0.5
                    loop_new = hillclimbing(loop, [7, 8, 9, 10], angle, 1)
                    residui_usati.append(1)
                    angoli_usati.append(angle)
                else:
                    loop_new = hillclimbing(loop, [7, 8, 9, 10], angle, n_res)
                    residui_usati.append(n_res)
                    angoli_usati.append(angle)
                start = distancefromstarting(loop_new, list_chain, 
                                        idx_chain_of_interest, coil_name[idx])
            else:
                n_res = suggestedresidue(len(loop), distanza_iniziale, 
                                        distanza_raggiunta)
                angle = suggestedangle(distanza_iniziale, distanza_raggiunta)
                swap = suggestedchange(distanza_iniziale, distanza_raggiunta)
                if(swap):
                    contatore_swap_effettuati += 1
                    if angolo_positivo:
                        angle = 0.5
                    else:
                        angle = -0.5
                    loop_new = hillclimbing(loop, [], angle, 1)
                    angoli_usati.append(angle)
                    residui_usati.append(1)
                else:
                    loop_new = hillclimbing(loop, [], angle, n_res)
                    residui_usati.append(n_res)
                    angoli_usati.append(angle)
                start = distancefromstarting(loop_new, list_chain, 
                                        idx_chain_of_interest, coil_name[idx])
            if (start):
                distance = objective_function(loop_new, 
                            residui_parte_mobile_after_rotation, coil_name[idx])
                distanza_raggiunta.append(distance)
                if (distance <= distanza_iniziale):
                    distanza_iniziale = distance
                    set_coordinate_loop(loop_new, idx)
                    distanza_migliore.append(distanza_iniziale)
                if (distanza_iniziale <= 0.1):
                    print("Ho raggiunto la distanza minima")
                    print("Mancavano tot iterazioni: ", iterazioni_massime-i)
                    break 
            
            get_coordinate_loop(loop, idx)

            if (i % print_pdb_timeout == 0):
                if (coil_name[idx] == 'A'):
                    insert_new_loop(list_chain, idx_chain_of_interest, loop, 
                                    costanti_loopA, coil_name[idx])
                else: 
                    loop_af = loop[::-1] 
                    insert_new_loop(list_chain, idx_chain_of_interest, loop_af, 
                                    costanti_loopB, coil_name[idx])
                title = "Situazione-"+coil_name[idx]+"-iter"+str(i)+".pdb"
                printpdb(title)
            
            if (stopsearch(distanza_iniziale, distanza_raggiunta)):
                print("Ho fermato la ricerca perchè non stavi   
                        migliorando abbastanza")
                #print(distanza_raggiunta)
                break

        print("Final distance: ", distanza_iniziale)
        ultime = distanza_raggiunta[len(distanza_raggiunta)-5:
                                    len(distanza_raggiunta)]
        print("Ultime 5 distanze: ", ultime)
        print("Swap effettuati: ", contatore_swap_effettuati)
        print('\n')

        title = "Loop-" + coil_name[idx] + ".csv"
        with open(title, 'w') as file:
            # 2. Create a CSV writer1
            writer = csv.writer(file)
            # 3. Write data to the file
            writer.writerow(header)
            for n in range(len(distanza_raggiunta)):
                if (n == 0):
                    data = [coil_name[idx], distanzeCA[n], scalar_product[n], 
                            distanza_raggiunta[n], usedweight[n], "None", 
                            "None", "None"]
                else:
                    data = [coil_name[idx], distanzeCA[n], scalar_product[n], 
                            distanza_raggiunta[n], usedweight[n], 
                            loop[residui_usati[n-1]].id[1], torsion_used[n-1], 
                            angoli_usati[n-1]]
                writer.writerow(data)
        
        if (coil_name[idx] == 'A'):
            insert_new_loop(list_chain, idx_chain_of_interest, loop, 
                            costanti_loopA, coil_name[idx])
            visualize_image(distanza_raggiunta, distanza_migliore, angoli_usati, 
                            residui_usati, "LoopA-distance")
        else:
            loop_af = loop[::-1] 
            insert_new_loop(list_chain, idx_chain_of_interest, loop_af, 
                            costanti_loopB, coil_name[idx])
            visualize_image(distanza_raggiunta, distanza_migliore, 
                        angoli_usati, residui_usati, "LoopB-distance")
\end{verbatim}
In questo secondo blocco di codice andiamo ad effettuare le operazioni per portare a convergenza i loop. La prima operazione da fare è quella di andare ad individuare 
la parte mobile, mediante una procedura che funziona in modo simile a quella descritta in precedenza per i loop. 

\begin{verbatim}
    def identificazione_parte_mobile(list_chain, chain_of_interest):
        mobil_part = []
        residue = list(list_chain[chain_of_interest].get_residues())
        for r in residue:
            if (r.id >= costanti_parte_mobile[0]) and 
                (r.id <= costanti_parte_mobile[1]):
                mobil_part.append(r)
        return mobil_part
\end{verbatim}
La procedura identificazione\_parte\_mobile si occupa di andare ad identificare la parte mobile soggetta poi a rotazione. Si sviluppa nello stesso modo rispetto alla 
procedura per l'identificazione dei loop. In questo caso non vengono passate costanti alla procedura perché la parte mobile è soltanto una quindi vengono prese dalle
variabili globali. In questo caso non ci sono aggiunte di residui in più per controllare la convergenza, ma solo ed esclusivamente i residui della parte mobile.
L'operazione successiva è quella di applicare una rotazione alla parte mobile.

\begin{verbatim}
    def apply_rotation_on_mobil_part(residui):
        rotation_radians = np.radians(rotation_angle)
        subtrac_coord, r_matrix = calcolocentroide(residui, rotation_radians)
        after_rotation = []
        for residuo in residui:
            for atom in residuo.get_atoms():
                atom_coord = atom.get_coord()
                ### Sottraggo le coordinate da sottrarre
                for k in range(len(atom_coord)):
                    atom_coord[k] = atom_coord[k] - subtrac_coord[k]
                    set_element(k, atom_coord[k])
                ### Dobbiamo moltiplicare
                for j in range(len(r_matrix)):
                    val = 0
                    for c1 in range(len(r_matrix[j])):
                        val += r_matrix[j][c1] * get_element(c1)
                    atom_coord[j] = val
                ### Sommiamo di nuovo le coordinate sottratte in precedenza
                for k in range(len(atom_coord)):
                    atom_coord[k] = atom_coord[k] + subtrac_coord[k]
                atom.set_coord(atom_coord)
            after_rotation.append(residuo)   
        return after_rotation
\end{verbatim}
La procedura apply\_rotation\_on\_mobil\_part ci permette di andare a ruotare di x gradi la parte mobile. Vengono trasformati in radianti i gradi di rotazione;viene 
calcolato il centroide della parte mobile come somma di tutte le coordinate diviso il numero di punti sommati viene normalizzato e viene la matrice di rotazione. Dopodiché
per rendere effettiva la rotazione si prendono le coordinate dei singoli atomi vi si sottraggono le coordinate del centroide, il risultato viene moltiplicato per la matrice
di rotazione e infine al risultato vengono sommate le coordinate sottratte in precedenza. Per evitare il problema di condivisione della memoria citato in precedenza mi 
appoggio ad un array globale con dei metodi di set e get riportati qui sotto. 
\begin{verbatim}
    def set_element(index, coord):
        array_appoggio[index] = 0
        array_appoggio[index] = coord

    def get_element(index):
        return array_appoggio[index]
\end{verbatim}

Oltre alla rotazione della parte mobile è possibile anche traslare la stessa secondo una direzione, il coefficente di traslazione e uno dei parametri presi da riga di 
comando e la traslazione avviene se e soltanto se il parametro è diverso da zero. 
\begin{verbatim}
    def applytransitionmove(residui):
        for residuo in residui:
            for atom in residuo.get_atoms():
                atom_coord = atom.get_coord()
                for i in range(len(atom_coord)):
                    if (transition_axis[i] == 1):
                        atom_coord[i] += transition_step
                atom.set_coord(atom_coord)
\end{verbatim}
La procedura applytransitionmove si occupa come detto in precendenza di applicare una traslazione secondo gli assi [x, y, z]. Viene preso ogni singolo atomo e per ciascuna
delle sue coordinate ci si domanda se è necessario traslarla, se necessario si aggiunge quindi lo step di traslazione. 

Dopo aver completato la rotazione e traslazione della parte mobile è necessario fare una copia della stessa per essere sicuri di non modificarla e in caso di modifica 
abbiamo sempre una copia della stessa. Dopodichè andiamo a reinserire la parte mobile modificata all'interno del file pdb attraverso la seguente procedura.

\begin{verbatim}
    def insert_new_mobil_part(list_chain, chain_of_interest, 
        residui_parte_mobile_after_rotation):
        residue = list(list_chain[chain_of_interest].get_residues())
        idx = 0
        for r in residue:
            if (r.id >= costanti_parte_mobile[0]) 
                and (r.id <= costanti_parte_mobile[1]):
                for atomr, atoml in zip(r.get_atoms(), 
                    residui_parte_mobile_after_rotation[idx].get_atoms()):
                    atomr.set_coord(atoml.get_coord())
                idx += 1
\end{verbatim}
La procedura insert\_new\_mobil\_part si occupa di andare ad inserire all'interno della struttura della proteina la parte mobile precedentemente modificata per fare ciò 
è necessario andare a scorrere di nuovo tutta la nostra catena di interesse e una volta trovati i residui corretti sostituire le loro coordinate con quelle nuove generate
dopo la rotazione. Dopo questa fase di lavoro sulla parte mobile si apre il capitolo sulla convergenza dei singoli loop, come detto in precedenza essi sono due e vengono
allineati in modo tale da non dover duplicare le operazioni. Come prima cosa dobbiamo andare a misurare quelle che sono le distanze di partenza.

\begin{verbatim}
    def objective_function(loop, mobil_part, loop_name):
        residuo_loop = None
        residuo_parte_mobile = None
        if (loop_name == 'A'):
            ### Devo calcolare la distanza tra l'ultimo elemento 
            ### del loop e il primo elemento della parte mobile
            residuo_loop = loop[len(loop)-1]
            residuo_parte_mobile = mobil_part[0]
        else:
            ### Devo calcolare la distanza tra l'ultimo elemento 
            ### del loop e il primo elemento della parte mobile
            residuo_loop = loop[len(loop)-1]
            residuo_parte_mobile = mobil_part[len(mobil_part)-1] 
        atom_loop = residuo_loop['CA']
        atom_ptmob = residuo_parte_mobile['CA']
        distanzaCA = atom_loop - atom_ptmob
        atomC_loop = residuo_loop['C'].get_coord()
        atomN_loop = residuo_loop['N'].get_coord()
        axis_coord_loop = atomN_loop - atomC_loop
        lenght = math.sqrt((axis_coord_loop[0] * axis_coord_loop[0]) + 
            (axis_coord_loop[1] * axis_coord_loop[1]) + 
            (axis_coord_loop[2] * axis_coord_loop[2]))
        for i in range(len(axis_coord_loop)):
            axis_coord_loop[i] = axis_coord_loop[i]/lenght

        atomC_ptmob = residuo_parte_mobile['C'].get_coord()
        atomN_ptmob = residuo_parte_mobile['N'].get_coord()
        axis_coord_ptmob = atomN_ptmob - atomC_ptmob
        lenght = math.sqrt((axis_coord_ptmob[0] * axis_coord_ptmob[0]) + 
            (axis_coord_ptmob[1] * axis_coord_ptmob[1]) + 
            (axis_coord_ptmob[2] * axis_coord_ptmob[2]))
        for i in range(len(axis_coord_ptmob)):
            axis_coord_ptmob[i] = axis_coord_ptmob[i]/lenght
        
        prodotto_scalare = 0
        for i, j in zip(range(len(axis_coord_ptmob)), range(len(axis_coord_loop))):
            prodotto_scalare += axis_coord_ptmob[i] * axis_coord_loop[j]
        weight = suggestedweight(distanzaCA)
        usedweight.append(weight)
        distanzeCA.append(distanzaCA)
        scalar_product.append(prodotto_scalare)
        distanza = distanzaCA + weight*(1-prodotto_scalare)
        return distanza
\end{verbatim}

La procedura objective\_function si occupa di calcolare la distanza tra due residui, ovvero nel caso del loopA l'ultimo del loop A il primo della parte mobile, mentre
nel caso del loop B l'ultimo della parte mobile. La distanza viene calcolata nel seguente modo ovvero, viene presa la distanza tra gli atomi C$\alpha$ dei due residui e
poi ci viene sommato il prodotto scalare generato dall'asse tra gli atomi N e C dei due residui, moltiplicato per un coefficente detto peso in modo da rendererlo influente.
Il prodotto scalare viene inserito per fare in modo che si converga nel modo corretto con la direzione corretta, altrimenti si rischia di non convergere perchè è possibile 
che N e C siano opposti rispetto all'altro residuo. Il peso che viene moltiplicato al prodotto scalare viene suggerito dalla procedura suggestedweight che ci restituisce
un peso in base alla distanza dei CA. Calcolata la distanza iniziale, ho pensato di applicare delle rotazioni "canoniche" dalla metà del loop in avanti sia su phi che su 
psi, questo ci garantisce di colmare grandi distanze immediatamente per poi lavorare di fino nella ricerca successiva. Le rotazioni "canoniche" si basano sui seguenti 
angoli: [-180, -90, -60, -30, -15, -5, -3, -1, -0.5, 0.5, 1, 3, 5, 15, 30, 60, 90, 180]. In precedenza ho nominato phi e psi che sono angoli torsionali, in verità applichiamo
la rotazione sull'asse dell'angolo torsionale, ovvero per quanto riguarda phi l'asse è tra C$\alpha$\-N, mentre psi l'asse è tra C$\alpha$\-C. La procedura che si occupa di 
applicare la rotazione è la seguente.

\begin{verbatim}
    def hillclimbing(loop, constat_to_not_touch, angle, 
                        n_res, direction = -1):
        loop_new = []
        ### Scegliamo a caso un atomo su cui agire
        idx_res_start = n_res
        ### Scegliamo se ruotare su phi o psi
        if (direction == -1):
            torsional_angle = random.randint(0,1)
        else:
            torsional_angle = direction
        torsion_used.append(torsional_angle)

        if(idx_res_start not in constat_to_not_touch):
            r_matrix =  []
            subtrac_coord = []
            if (torsional_angle == 0):
                atomN_coord = None
                atomCA_coord = None
                for atom in loop[idx_res_start].get_atoms():
                    if atom.id == 'N':
                        atomN_coord = atom.get_coord()
                    if atom.id == 'CA':
                        atomCA_coord = atom.get_coord()
                ### Vettore u su cui dobbiamo ruotare
                subtrac_coord = atomN_coord
                axis_coord = atomCA_coord - atomN_coord
                r_matrix = rotation_matrix(axis_coord, angle)
            else:
                ### Dobbiamo Calpha - C
                atomC_coord = None
                atomCA_coord = None
                for atom in loop[idx_res_start].get_atoms():
                    if atom.id == 'C':
                        atomC_coord = atom.get_coord()
                    if atom.id == 'CA':
                        atomCA_coord = atom.get_coord()
                ### Vettore u su cui dobbiamo ruotare
                subtrac_coord = atomC_coord
                axis_coord = atomCA_coord - atomC_coord
                r_matrix = rotation_matrix(axis_coord, angle)
            
            ### Con questo primo ciclo copio semplicemente 
            i vecchi residui che non verranno modificati
            for i in range(0, idx_res_start):
                loop_new.append(loop[i])
        
            ### Con questo secondo ciclo applico la rotazione 
            sugli atomi
            
            for i in range(idx_res_start, len(loop)):
                if (torsional_angle == 0):
                    if i == idx_res_start:
                        for atom in loop[i].get_atoms():
                            if (atom.id != 'N' and atom.id != 'CA'):
                                atom_coord = atom.get_coord()
                                ### Sottraggo le coordinate da sottrarre
                                for k in range(len(atom_coord)):
                                    atom_coord[k] = atom_coord[k] - 
                                                subtrac_coord[k]
                                    set_element(k, atom_coord[k])
                                ### Dobbiamo moltiplicare
                                for j in range(len(r_matrix)):
                                    val = 0
                                    for c1 in range(len(r_matrix[j])):
                                        val += r_matrix[j][c1] * 
                                            get_element(c1)
                                    atom_coord[j] = val
                                ### Sommiamo di nuovo le coordinate sottratte 
                                ### in precedenza
                                for k in range(len(atom_coord)):
                                    atom_coord[k] = atom_coord[k] + 
                                        subtrac_coord[k]
                                atom.set_coord(atom_coord)
                    else:
                        for atom in loop[i].get_atoms():
                            atom_coord = atom.get_coord()
                            ### Sottraggo le coordinate da sottrarre
                            for k in range(len(atom_coord)):
                                    atom_coord[k] = atom_coord[k] - 
                                        subtrac_coord[k]
                                    set_element(k, atom_coord[k])
                            ### Dobbiamo moltiplicare
                            for j in range(len(r_matrix)):
                                val = 0
                                for c1 in range(len(r_matrix[j])):
                                    val += r_matrix[j][c1] * 
                                        get_element(c1)
                                atom_coord[j] = val
                            ### Sommiamo di nuovo le coordinate sottratte 
                            ### in precedenza
                            for k in range(len(atom_coord)):
                                atom_coord[k] = atom_coord[k] + 
                                    subtrac_coord[k]
                            atom.set_coord(atom_coord)
                if (torsional_angle == 1):
                    if i == idx_res_start:
                        for atom in loop[i].get_atoms():
                            if (atom.id != 'N' and atom.id != 'C' 
                                and atom.id != 'CA'):
                                atom_coord = atom.get_coord()
                                ### Sottraggo le coordinate da sottrarre
                                for k in range(len(atom_coord)):
                                    atom_coord[k] = atom_coord[k] - 
                                        subtrac_coord[k]
                                    set_element(k, atom_coord[k])
                                ### Dobbiamo moltiplicare
                                for j in range(len(r_matrix)):
                                    val = 0
                                    for c1 in range(len(r_matrix[j])):
                                        val += r_matrix[j][c1] * 
                                            get_element(c1)
                                    atom_coord[j] = val
                                ### Sommiamo di nuovo le coordinate sottratte 
                                ### in precedenza
                                for k in range(len(atom_coord)):
                                    atom_coord[k] = atom_coord[k] + 
                                        subtrac_coord[k]
                                atom.set_coord(atom_coord)
                    else:
                        for atom in loop[i].get_atoms():
                            atom_coord = atom.get_coord()
                            ### Sottraggo le coordinate da sottrarre
                            for k in range(len(atom_coord)):
                                atom_coord[k] = atom_coord[k] - 
                                    subtrac_coord[k]
                                set_element(k, atom_coord[k])
                            ### Dobbiamo moltiplicare
                            for j in range(len(r_matrix)):
                                val = 0
                                for c1 in range(len(r_matrix[j])):
                                    val += r_matrix[j][c1] * 
                                        get_element(c1)
                                atom_coord[j] = val
                            ### Sommiamo di nuovo le coordinate sottratte 
                            ### in precedenza
                            for k in range(len(atom_coord)):
                                atom_coord[k] = atom_coord[k] + 
                                    subtrac_coord[k]
                            atom.set_coord(atom_coord)
                loop_new.append(loop[i])
        else: 
            loop_new = loop
        return loop_new
\end{verbatim}

La procedura hillclimbing si occupa di andare ad applicare una rotazione su uno degli assi degli angoli torsionali (phi, psi). Alcuni parametri hanno dei valori di 
default che cambiano quello che è il comportamento della procedura, ovvero l'angolo torsionale può essere deciso sia randomicamente all'interno della procedura oppure
può essere passato dall'esterno come nel caso delle rotazioni "canoniche" in cui ogni singolo angolo viene provato sia per phi che psi. Una volta deciso l'angolo 
torsionale si verifica che la scelta del residuo su cui vogliamo cominciare ad effettuare le rotazioni non compaia dentro a quei residui esplicitati nella variabile
constat\_to\_not\_touch e poi si compie la rotazione. Ci sono residui che non devono essere toccati ovvero, che da quei residui non può partire nessuna rotazione 
perché sono una parte fissa, mentre se la rotazione parte prima loro possono comunque essere coinvolti. La rotazione su phi o psi non cambia il metodo, ma cambio solo 
gli interpreti della rotazione. 

Terminate le rotazioni "canoniche" si cominciano le iterazioni specificate come parametro del programma e in queste rotazioni il residuo e l'angolo su cui si va a
lavorare vengono suggeriti dalle seguenti procedure.
\begin{verbatim}
    def suggestedangle(distanza, dist_ragg):
        global angolo_positivo
        global angolo_positivo_residuo
        if (len(dist_ragg)-angolo_positivo_residuo > 49):
            change = False
            for i in range(len(dist_ragg)-50, len(dist_ragg)):
                if(dist_ragg[i] > distanza):
                    change = True
                else:
                    change = False
                    break
            if (change):
                angolo_positivo_residuo = len(dist_ragg)
                angolo_positivo = not(angolo_positivo)
        angle = 0

        if (distanza >= 20):
            angle = 5
        elif (distanza >= 15):
            angle = 4
        elif (distanza >= 10):
            angle = 3
        elif (distanza >= 5):
            angle = 2
        elif (distanza >= 2):
            angle = 1
        elif (distanza >= 1):
            angle = 0.7
        elif (distanza >= 0.5):
            angle = 0.5
        else:
            angle = 0.1

        if angolo_positivo:
            return angle
        else:
            return -angle

    def suggestedresidue(lunghezza, distanza, dist_ragg):
        n_res = lunghezza-1
        n_res = random.randint(1, lunghezza-1)
        return n_res
\end{verbatim}

La procedura suggestedangle mi suggerisce un angolo e il segno dell'angolo ovvero l'angolo viene deciso mediante confronto della distanza, mentre il segno viene 
deciso in base ai risultati dell'esecuzione. La distanza considerata è la migliore ottenuta fino in questo momento, mentre vengono prese le ultime 50 distanze e sono
tutte maggiori rispetto alla distanza migliore si inverte il precedente segno dell'angolo, in modo alternare. 

La procedura suggestedresidue mi suggerisce in modo randomico un residuo su cui andare a lavorare, viene escluso il primo perchè è per entrambi quello di partenza. 
Durante l'esecuzione delle iterazioni potrebbe essere necessario eseguire una grossa rivoluzione, ovvero ho pensato ad una tecnica per ribaltare il risultato.
\begin{verbatim}
    def suggestedchange(distanza, dist_ragg):
        swap = False
        global swap_residuo
        if (len(dist_ragg)-swap_residuo  > 10):
            for i in range(len(dist_ragg)-11, len(dist_ragg)):
                if (dist_ragg[i] > distanza and dist_ragg[i] <= distanza+0.5):
                    swap = True
                else:
                    swap = False
                    break
        if (swap):
            swap_residuo = len(dist_ragg)
        return swap
\end{verbatim}
La procedura suggestedchange è usata per suggerire al programma principale di applicare un cambiamento ovvero se per 10 risultati consecutivi dell'esecuzione otteniamo
una distanza risultante compresa tra la distanza migliore e la distanza migliore a cui viene aggiunto 0.5 significa che non risciamo a trovare una via per avvicinarci di 
più quindi applicare una rotazione alla base del loop di mezzo grado potrebbe darci una mano per ottenere il nostro risultato. 

Una volta terminata la rotazione verifichiamo che non ci siamo discostati dal punto di partenza mediante la funzione distancefromstarting, che non fa altro che verificare
che le distanze presenti in precedenza siano ancora invariate, misurariamo quindi la distanza del loop e se ci siamo avvicinati ci teniamo il nuovo risultato e il nuovo 
loop altrimenti torniamo alla versione precedente e ricominciamo da li. 

Ci sono due modi per interrompere preventivamente l'esecuzione del programma: raggiungere la distanza minima di 0.1; oppure essere fermati da una procedura che verifica
che negli ultimi 1000 passaggi abbiamo fatto progressi per almeno 0.01.