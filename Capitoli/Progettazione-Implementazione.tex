\chapter{Progettazione}\label{chapter:progettazione}
%ecco qualche  commento.
%la sezione 4 richiede un po' di lavoro.
%manca tutto il livello di spiegazione tra l'introduzione e la sezione 4 attuale.
%in pratica devi raccontare in modo sintetico le strategie scelte (senza troppi dettagli
%implementativi), come hai organizzato le strutture di supporto
%e il perche'.
%Poi puoi raccontare lo pseudo codice, ma non nello stato attuale (riorganizzalo, semplificalo).
%Ci sta anche qualche disegno astratto sulla strategia di local search (puoi fare una griglia 2d
%che mostra il vicinato di traslazione o cose simili) ---> ?

%%% Sezione in cui spiegavo l'uso del biopython, causato dai file in ingresso che sono dei pdb
%%% Sezione sulle tecniche utilizate e le strutture dati che mi hanno aiutato all'interno del programma
%%% Poi avendo un approccio top-down spiego lo speudo codice
In questo capitolo viene affrontata la parte di implementazione e progettazione, vedremo le strategie di ricerca scelte, l'organizzazione delle strutture dati a supporto dell'esecuzione e poi avremo un approccio top-down verso il codice scritto. Introduco anche la libreria di supporto che ci ha permesso di svolgere il lavoro sul python.

\section{Biopython}\label{sec:Biopython}
Il progetto Biopython è una libreria opensource \cite{BioPythonManual} di strumenti python per la biologia computazionale e la bioinformatica. Essa contiene classi per rappresentare sequenze biologiche e annotazioni di sequenze ed è in grado di leggere e scrivere file provenienti da diversi formati. 
Questa libreria mi ha permesso di utilizzare il parser per i file pdb (Protein Data Bank) al cui interno sono codificate le informazioni riguardanti gli atomi come nome dell'atomo posizione all'interno dello spazio tridimensionale ed altre informazioni. Oltre a fornirmi un parser, al suo interno sono presenti dei moduli che permettono di generare oggetti come le catene, i residui e gli atomi stessi.
Questo da la possibilità di chiamare metodi specifici.  

\section{Strategie di ricerca}\label{sec:Strategiediricerca}
All'interno di questo framework è stato necessario usare tecniche di ricerca locale nella fase in cui si cerca di andare a ristabilire il collegamento dei due loop alla parte mobile una volta effettuata una rotazione/traslazione su di essa. In questo caso la tecnica utilizzata è la hill-climbing, in cui si cerca di trovare il massimo (o il minimo) di una funzione di costo attraverso la ricerca di soluzioni vicine a quella corrente. In questo caso specifico, la funzione di ricerca cerca di ottimizzare la conformazione di un loop di collegamento alla volta attraverso la rotazione di alcuni angoli torsionali. Si parte da una configurazione iniziale del loop rappresentata mediante ad una lista di residui, e cerco di migliorarla ruotando uno dei due angoli torsionali a disposizione, partendo da un residuo specificato. L'angolo torsionale viene scelto casualmente tra $\phi o \psi$. Una volta applicata la rotazione viene restituita la nuova conformazione della proteina e si procede in questo modo fin tanto che non viene raggiunto il risultato desiderato. 

Viene utilizzata anche una tecnica di ricerca per guidare il processo che porta al completamento del percorso più breve tra le due configurazioni. La strategia di ricerca, in questo caso, è una forma di shortest path con una coda di priorità implementata tramite heap, albero binario ordinato. Essa viene utilizzata per selezionare il nodo con il costo totale (cioè il costo del percorso finora più la stima del costo rimanente) più basso in ogni iterazione. In questo modo si cerca di espandere i nodi che hanno costo totale più basso per primi. 

\section{Strutture dati a supporto}\label{sec:Strutturedati}
Le strutture dati sono importanti all'interno di un algoritmo, non solo per permettere di conservare informazioni, ma incidono anche sul tempo di esecuzione del singolo programma. Organizzare quindi l'accesso alle strutture dati nel modo migliore permette di risparmiare tempo d'esecuzione oltre a semplificare nella maggior parte dei casi la comprensione della stessa. 

Il linguaggio di programmazione scelto, python, necessità di studiare nel modo corretto come accedere alle strutture dati per evitare di trovarsi con dati inconsistenti poiché sono frutto di più modifiche. Per questo motivo e per quanto detto in precedenza io ho previsto 8 strutture di dati importanti: due dedicate a contenere la parte mobile e i loop, due dedicate al salvataggio delle posizioni acquisite da parte mobile e loop durante il cammino da una configurazione all'altra, due dedicate al salvataggio temporaneo e al ripristino in caso di clash delle coordinate dei loop pre e post rotazione e poi due dedicate alla gestione dei clash.

Le strutture dati sono state quasi tutte sviluppate come dei dizionari ad accesso chiave valore. Per quanto riguarda la struttura dati dedicata al controllo della presenza dei clash, la struttura contiene tuple di coordinate tridimensionali e per ciascuna tupla è contenuto la catena il residuo e l'atomo che si trovano in quella posizione. Questo è stato possibile mediante una discretizzazione di tutti gli atomi all'interno di una matrice tridimensionale. Chiaramente il passo di campionamento può essere impostato in qualunque modo, di default è impostato a 5 Ångström. \footnote{Ångström, sono un'unità di misura, non SI, pari a $10^-10m$ è molto utilizzata nell'ambito molecolare.} Per quanto riguarda le strutture dati per mantenere salvate le coordinate dei loop e della parte mobile anch'esse sono organizzate come dizionari la cui chiave indicizza l'insieme delle varie coordinate. Le chiavi che indicizzano la struttura dati sono delle tuple contenenti la tripletta di traslazione effettuata più un numero in più che codifica il grado di rotazione della parte mobile. Attraverso metodi di get e set possiamo settarli nelle variabili coinvolte nel processo evitando problemi di condivisione della memoria.
Abbiamo anche due dizionari dedicati al salvataggio temporaneo e al ripristino delle coordinate nel caso la rotazione non sia buona. Questi dizionari sono solamente per i due loop all'interno della procedura che si occupa di farli convergere alla parte mobile. La struttura dedicata ai loop e alla parte mobile invece sono due liste che a loro volta contengono residui a cui applicare le varie rotazioni. Voglio comunque porre enfasi sullo studio necessario per progettare la singola struttura che mi ha concesso non solo di risolvere il problema della condivisione della memoria, ma anche di velocizzare il processo di accesso alla struttura dati e quindi portare più velocità nell'esecuzione della ricerca locale.


\section{Approccio Top-Down al codice}\label{sec:ApproccioTopDown}
Ora vediamo quello che è lo speudo-codice che permette di compiere il lavoro all'algoritmo. 

\begin{minipage}{\textwidth}
	\begin{lstlisting}[caption={Fase d'inizializzazione del framework}]
		parser = PDBParser(PERMISSIVE = True, QUIET = True)
		covidchiusa = parser.getstructure(titleclosed)
		covidaperta = parser.getstructure(titleopen)
		calcoloparallelepipedo(covidchiusa, covidaperta)
		loop = identifyloop(listacatene, catenainteresse, costantiA)
		coil.append(loop)
		loop = identifyloop(listacatene, catenainteresse, costantiB)
		loopr = loop[::-1]
		coil.append(loopr)
		
		coildictcoordinates = [{"loop": x, "residui":[{"residuo":y, "coordinate": []} for y in range(len(coil[coilname.index(x)]))]} for x in coilname]
		
		for loop,idx in zip(coil, range(len(coil))):
			salviamo(coildictcoordinates, loop)
			salviamo(coildictrestorecoordinates, loop)
		
		ptmob = partemobile(listacatene, catenainteresse)
		storedptmob = salviamo(ptmob)
		if (key not in posizioniacquisiteptmob):
			insert(posizioniacquisite, key)
		
		residuipartefissa = partefissa(listacatene, catenainteresse)
		coordinate = salviamo(coildictcoordinates)
		if (key not in posizioniacquisiteloop):
			insert(posizioniacquisiteloop, key)
	
	\end{lstlisting}
\end{minipage}
Come si può vedere all'interno di questo primo spezzone di pseudo-codice si vanno ad inizializzare le componenti principali del sistema. Innanzitutto viene creato un parser che permette ha il compito di parsare il file pdb in ingresso. Dopodiché si effettua la lettura sia della configurazione chiusa che di quella aperta, questo è necessario perché come si vede nel prossimo passo dobbiamo generare un parallelepipedo che contenga entrambe le parti mobili. 

\begin{minipage}{\textwidth}
	\begin{lstlisting}[caption={Calcolo del parallelepipedo}]
		def calcoloparallelepipedo(chiusa, aperta):
		global variabili globali
		x = []
		y = []
		z = []
		mobilclosedpart = partemobile(listacatene, catenainteresse)
		mobilopenedpart = partemobile(listacatene, catenainteresse)
		model_open = aperta.get_models()
		
		for mobil in [mobilclosedpart, mobilopenedpart]:
			for residue in mobil:
				for atom in residue.getatom():
					diviamo le coordinate nei gli array degli assi cartesiani corrispondenti
		
		for coordinata in [x, y, z]:
			for indicatore in [max, min]:
				assegnamo alla variabile globale del parallelepipedo la coordinata corretta

	\end{lstlisting}
\end{minipage}

Il parallelepipedo che contiene entrambe le parti mobili è necessario per restringere il campo di ricerca, questo perché trovandoci in spazio aperto è fondamentale limitare lo spazio di ricerca ad una precisa zona altrimenti il processo non terminerebbe mai. 
Come detto nel capitolo \ref{chapter:obbiettivi} si lavora sulla configurazione chiusa e si tenta di convergere nella configurazione aperta, quindi tutto ciò che viene eseguito di seguito avviene sulla configurazione chiusa; la configurazione aperta è quindi solo necessaria per capire dove ci dirigiamo con la ricerca. Procedendo si passa ad individuare i loop, che mettono in collegamento la parte mobile con la parte fissa. I due loop sono simili, ma diversi nella direzione da loro presa, ovvero il loop-A si dirige dalla parte fissa alla parte mobile, mentre il loop-B lavora al contrario quindi per rendere omogeneo il comportamento del programma nei confronti dei due loop è necessario capovolgere il secondo. Questo non comporta nessun altra modifica, è però necessario ricordarsi di capovolgere il loop nel momento in cui si effettua il processo di scrittura file. Le costanti sul posizionamento dei loop sono state fornite insieme alle configurazioni ed è importante sottolineare che è stato preso un residuo in più in capo e in coda al loop, per assicurarsi che ogni modifica si adattasse bene alla struttura a cui si deve attaccare. 
Si intravede l'organizzazione di una delle strutture citate in \ref{sec:Strutturedati}, essa è necessaria per salvare temporaneamente le coordinate dei loop, in modo tale che l'effetto di una rotazione può essere testato prima di diventare definitivo. Dopodiché si passa alla ricerca della parte mobile e della parte, in questo caso non sono necessarie le costanti, ma per identificarle andiamo ad esclusione rispetto a quelle utilizzate per i loop. Si intravedono anche le strutture dati dove sono conservati tutti gli step che portano da una configurazione all'altra. 

\begin{minipage}{\textwidth}
	\begin{lstlisting}[caption={Modalità di esecuzione}]
		if len(sys.argv) != 5 and len(sys.argv) > 1:
			print("Devi passare cinque parametri allo script!")
			sys.exit()
		else:
			iterazionimassime = 15000
			printpdbtimeout = 25000
			step = 5   
			if (len(sys.argv) == 5):
				print("RICERCA SOLUZIONE SPECIFICA") 
			else: 
				if (len(sys.argv) == 1):
					print("RICERCA DI POTENZIALI NUOVE SOLUZIONI")
					costo, listachiavi = shortestpath((0,0,0,0))
					for chiave in listachiavi:
						Recuperiamo le coordinate e le stampiamo in un file di output				
	\end{lstlisting}
\end{minipage}

Il framework può essere utilizzato in due modalità come si può vedere dal codice. In una modalità si può ricercare una soluzione specifica, ovvero è possibile dare un angolo di rotazione alla parte mobile una traslazione e il programma nel numero di iterazioni consentite cercherà la convergenza dei loop. Nell'altra modalità si va invece a cercare lo shortest-path tra le due configurazioni. Una volta terminata la ricerca del percorso vengono ricaricate tutte le posizioni acquisite e per ognuna di esse viene creato un file pdb apposito. Terminato il processo grazie all'uso di un altro script a disposizione è possibile generare un video per avere una prova grafica di cosa è accaduto durante la ricerca. 